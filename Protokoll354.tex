\documentclass[11pt,ngerman,a4paper]{article}
%Gummi|061|=)
\usepackage{amsmath}
\usepackage{a4wide}
\usepackage{amsthm}
\usepackage{amsbsy}
\usepackage{amssymb}
\usepackage{inputenc}
\usepackage{rotating} 
\usepackage{graphicx}
\usepackage{paralist}
\usepackage{selinput}
\SelectInputMappings{%
adieresis={ä},
germandbls={ß},
}
\title{\textbf{Versuch V355: Gekoppelte Schwingkreise}}
\author{Martin Bieker\\
		Julian Surmann\\
		\\
		Durchgef\"{u}hrt am 09.01.2014\\
		Tu Dortmund}
\date{}
\usepackage{graphicx}
\begin{document}
\renewcommand\tablename{Tabelle}
\renewcommand\figurename{Abbildung}
\maketitle
\thispagestyle{empty}
\newpage
\clearpage
\setcounter{page}{1}


\section{Einleitung}
Der folgende Versuch gekoppelte Oszillatoren untersucht werden. Bei diesen Systemen handelt es sich um schwingfähige Systeme, welche miteinander in Welchselwirkung stehen und so Energie austauschen können. 

\section{Theorie}
\begin{equation}
\nu^+ = \frac{1}{2 \pi \sqrt{LC}}
\end{equation}

\begin{equation}
\nu^- =\frac1{2\pi\sqrt{L(\frac1C + \frac2{C_K})}}
\end{equation}


\section{Aufbau und Durchf\"{u}hrung}
\subsection{Vorbereitungen}
Vor Beginn der Messungen müssen die Resoanzfrequnezen beider Schwinkreise aufeinander abgestimmt werden. Daher ist in einem der Kreise ein Kondensator mit variabler Kapazität verbaut. 
\subsection{Messung vom Schwebungs- und Schwingungsfrquenz}
\subsection{Messung der Frequenzen der Fundamentalschwingungen}
\subsection{Messung des Stroms in Abhängigkeit von der Erregerfrequenz}
\section{Auswertung}

\section{Diskussion}
Sofern nicht anders angegeben, wurden alle Fehler mit python uncertainties berechnet.
\subsection{Vorbereitungen}
Für die Justierung des Messaufbaus wurde mit der Grobmessung eine Frequenz von $(30.69\pm0.01)\,kHz$ ermittelt. Die feine Messung mit Hilfe der Lissajous-Figuren ergab eine Resonanzfrequenz von $(30.70\pm0.01) \, kHz$. Der mit L und C theoretisch berechnete Wert liegt bei $30.49 \, kHz$, mit einer Abweichung von $0.68 \%$ ausgezeichnet im Toleranzbereich. Allerdings wurde in der Berechnung des theoretischen Wertes die Kapazität der Spule schon berücksichtigt.
\subsection{Messung von Schwebungs- und Schwingungsfrequenz}
Die Perioden pro Schwebungsperiode wurden so genau wie möglich abgelesen. Der Fehler liegt bei maximal $\pm 0.5$ Perioden. Die ermittelten Werte sind der Tabelle EINFÜGEN zu entnehmen.


\subsection{Messung der Frequenzen der Fundamentalschwingungen}
\subsection{Messung des Stroms in Abhängigkeit von der Erregerfrequenz}

\section{Literatur- und Abbildungsverzeichnis}
\begin{itemize}
\item $[1]$:Quelle
\end{itemize}
\section{Anhang}
\begin{itemize}
\item Tabellen
\item Auszug aus dem Messheft


\end{itemize}

\newpage
\end{document}